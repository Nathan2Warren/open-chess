\documentclass[architecture.tex]{subfiles}
\begin{document}
This document outlines the software architecture for the 
web-based application \textit{Open Chess} that offers
an interactive chess board through a React Application
served with Flask. 
Using the application (playing a game of chess through mouse interaction),
the user is encouraged to explore the early-game strategies of the game of chess.
The backend of the application (written in Flask, with a MongoDB database) 
internally evaluates positions and assigns them scores. 
The evaluation process is performed by Stockfish, a chess computer engine. 
Stockfish is used to assign scores to board positions fed to it, 
representing a move's soundness in the game. 
The scores for the initial and resulting board states in a move
form the weight of the move, taking into account 
other avaliable moves in the initial position.

The application defines \textit{theory}, which are moves in certain
positions of the board that are known to be good, typically generated by
parsing a pre-compiled set of games in the Polyglot format, or by reading
in PGN files of Grandmaster games.
Moves that are known by the application are shown as arrows color-coded by their
score: A good move is green, a bad move is red, and a theory move is blue.
Their score as evaluated by Stockfish are available as arrow tooltips.
Moves that have been added (through parsing Polyglot or PGN) but not yet evaluated
are white.

Making a move, by dragging and dropping the pieces in the frontend client,
updates the position with a fetch call to the backend. Making an unknown move
will trigger an analysis of the position, showing a loading spinner during
the few seconds of analysis time.

The app is intended for use as a training tool, prompting the player
to make good moves in pre-determined board positions.
This documentation will showcase the features of the software and
its modes of usage, along with a technical description of the database
and client-server communication.
The Open Chess source code is on \url{https://github.com/WilliamStenberg/open-chess},
and is Free Software.


\end{document}
