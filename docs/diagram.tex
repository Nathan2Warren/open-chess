\documentclass[a4paper,11pt,x11 names]{article}
\usepackage{tikz}
\usepackage{hyperref}
\usetikzlibrary{er}
\tikzset{multi attribute/.style={attribute ,double distance =1.5pt}}
\tikzset{derived attribute/.style={attribute ,dashed}}
\tikzset{total/.style={double distance =1.5pt}}
\tikzset{every entity/.style={draw=orange , fill=orange!20}}
\tikzset{every attribute/.style={draw=MediumPurple1, fill=MediumPurple1!20}}
\tikzset{every relationship/.style={draw=Chartreuse2, fill=Chartreuse2!20}}
\newcommand{\key}[1]{\underline{#1}}
\begin{document}
\section*{Database layout}
\begin{tikzpicture}[node distance =7em]
    \node[entity] (board) {Board};
    \node[attribute] (zobrist) [left of=board, xshift=-2em] {\key{Zobrist Hash}} edge (board);
    \node[attribute] (score) [above left of=board] {Score} edge (board);
    \node[attribute] (time) [above of=board] {Engine Time} edge (board);

    \node[relationship] (cando) [above right of=board, xshift=2em, yshift=-2em] {Can do} edge (board);

    \node[entity] (move) [right of=board, xshift=7em] {Move} edge (cando);
    \node[attribute] (uci) [above of=move] {UCI} edge (move);
    \node[attribute] (san) [above right of=move] {SAN} edge (move);
    \node[attribute] (mid) [right of=move] {\key{ID}} edge (move);
    \node[derived attribute] (weight) [below right of=move] {Weight} edge (move);

    \node[relationship] (leadsto) [below right of=board, xshift=2em, yshift=2em] {Leads to} edge (move);
    % Redeclaration for edge reference
    \node[entity] (board) {Board} edge (leadsto);

    \node[relationship] (fave) [below of=board] {Favourite} edge (board);
    \node[entity] (player) [below right of=fave, xshift=2em] {Player} edge (fave);
    \node[attribute] (pname) [below right of =player] {\key{Name}} edge (player);

    \node[relationship] (knows) [below of=move] {Knows} edge (player);
    \node[relationship] (knows) [below of=move] {Knows} edge (move);
    \node[attribute] (rank) [below right of=knows] {Rank} edge (knows);

\end{tikzpicture}
Above is an entity-relationship diagram displaying the database layout
of \textit{open-chess}.
The method of drawing the diagram comes from 
\url{https://www.guitex.org/home/images/ArsTeXnica/AT015/drawing-ER-diagrams-with-TikZ.pdf}.
Below, each of the elements are explained in terms of their intended
use in the application:
\begin{itemize}
    \item \textbf{Board}: Representing a chess board, i.e. a configuration
        of pieces. The pieces themselves are not modelled in the database,
        but during a session an active \textit{python-chess} Board
        object will hold such information.

    \item \textbf{Zobrist Hash}: A chess board is uniquely defined
        by its Zobrist hash (TODO: cite Zobrist theory),
        which is computed by taking into account piece positions,
        castling rights etc, resulting in a hash string.
    \item \textbf{Score}: A board's score is its evaluation by a chess
        engine - who's winning. A score is computed from a player's 
        perspective to a custom format,
        but this can be converted to a numeric value, using negative
        values for black's advantage and positive for white's.
    \item \textbf{Engine Time}: Since the score of a position is computed
        by an engine (Stockfish) and this score is more precise the longer
        the engine was allowed to think, the time that the engine has been
        run is kept so that scores can be improved. When simply loading
        in boards from an external source, this is set to zero
        indicating that any score cannot be relied upon.
    \item \textbf{Can do}: From a board there is a set of legal moves
        that can be made from that position. Since the board state
        implicitly convey whose turn it is to move, a board will
        only be connected to moves for a certain side (white or black).
    \item \textbf{Move}: Representing a legal move in a game of chess. 
    \item \textbf{UCI}: The UCI representation of the move, 
        comprised of the square the moved piece starts and lands on,
        concatenated to a four-letter string, e.g. ``e2e4'', ``b1c3''.
    \item \textbf{SAN}: The SAN representation of the move, which
        is more commonly used by humans when talking about moves,
        e.g. ``e4'', ``Nc3''. The SAN representation implicitly holds
        board state information.
    \item \textbf{ID}: Since two moves may have the same SAN and UCI
        representations, but pertain to different board states,
        a unique ID is required. Typically a numerical index.
    \item \textbf{Weight}: A derived property conveying how good
        a move is for the side making it, computed from the scores
        of the initial and resulting boards. A negative value would indicate
        a bad move, while a positive move is better since it improves the 
        board score for the move-making side.
    \item \textbf{Leads to}: The relationship between a move and the board
        it leads to. While a board \textit{can do} many moves, a move
        only \textit{leads to} one board.
    \item \textbf{Player}: A registered user of the system, created
        when they enter the application interface for the first time.
    \item \textbf{Name}: A player is uniquely identified by a string.
    \item \textbf{Favourite}: A player may record board positions
        to be remembered, to quickly place the interface's board in this
        state.
    \item \textbf{Knows}: A player is said to know a move if they have
        performed it in the interface. When a player knows a move,
        the interface's \textit{game mode} expects the player to be able
        to perform it if the session's current board \textit{can do} it.
    \item \textbf{Rank}: Each time the game mode prompts the player for a move
        in a board where the player knows at least one move, the player's action
        will influence the rank of one or several moves:
        If the player performs a known move, the rank is improved.
        If the player performs another move (not known by the player),
        the rank of all the known moves for that board will decrease.
        It is thus a measure of the player's familiarity the the known move.
\end{itemize}


\end{document}
