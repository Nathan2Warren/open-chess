\documentclass[architecture.tex]{subfiles}
\begin{document}

\begin{figure}[H]
\subfile{diagram.tex}
\caption{Entity-Relationship (ER) diagram for \textit{open-chess} database}
\label{fig:er-diagram}
\end{figure}

The entity-relationship diagram in Figure \ref{fig:er-diagram} outlines the structure of the 
database, implemented in SQLAlchemy in the backend of \textit{open-chess}.
The method of drawing the diagram comes from 
\url{https://www.guitex.org/home/images/ArsTeXnica/AT015/drawing-ER-diagrams-with-TikZ.pdf}.
Below, each of the elements are explained in terms of their intended
use in the application:

\begin{itemize}
    \item \textbf{Board}: Representing a chess board position, i.e. a configuration
        of pieces. The board is populated with known moves, and when analysed
        also a score. When adding games by PGN, boards can also hold information
        about what games they were seen in. This is interesting to those trying
        to study certain strong players.
        \begin{itemize}
            \item \textbf{FEN} (index): A board is uniquely defined
                by its FEN string (Forsyth-Edwards Notation)
                which is computed by taking into account piece positions and encodes
                which player is to move next.
                This is used as the index, allowing for quick lookups.
            \item \textbf{Score}: A board's score is its numeric evaluation by a chess
                engine - who's winning. A score is computed from a player's 
                perspective to a centipawn (hundredths of a pawn-up-advantage).
                A positive score with White to move indicates their advantage,
                as does a positive score for Black with Black to move.
            \item \textbf{Moves}: An array of objects representing legal moves from the
                board position. Contains the move's UCI, SAN, score difference,
                and the board the move leads to, by FEN.
            \item \textbf{Theory}: An array of Move objects that are separated to
                be identified as theory moves, verified by high-level play.
            \item \textbf{Games}: A list of ObjectIDs to Game objects, see below.
        \end{itemize}
    \item \textbf{Move}: Not in its own collection in the database, these
        objects are used to transition between boards.
        \begin{itemize}
            \item \textbf{UCI}: The UCI representation of the move, 
                comprised of the square the moved piece starts and lands on,
                concatenated to a four-letter string, e.g. ``e2e4'', ``b1c3''.
            \item \textbf{SAN}: The SAN representation of the move, which
                is more commonly used by humans when talking about moves,
                e.g. ``e4'', ``Nc3''. 
            \item \textbf{Score difference}:
                Centipawn difference between board positions before and after the move.
                A positive score difference indicates a sound move, while a negative
                means the move might be a mistake or even a blunder.
            \item \textbf{Leads to}: FEN of the resulting board after the move is made;
                The relationship between a move and the board it leads to.
                This is used to traverse the tree of moves and boards,
                using lookings on the Board collection by FEN.
        \end{itemize}
    \item \textbf{Game}: A subset of PGN game information. Contains the black
        and white players and their respective Elo ratings, and the outcome
        of the game as a string out of ``1 - 0'', ``0 - 1'', and ``1/2 - 1/2''.
\end{itemize}


\end{document}
